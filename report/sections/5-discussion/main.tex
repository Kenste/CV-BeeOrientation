We demonstrated that head/tail segmentation could be employed to estimate bee orientation to a certain degree. However, we did not achieve the level of orientation accuracy reported by Bozek et al.~\cite{bozek2021markerless}.

ResUNet18 produced slightly lower orientation errors and better segmentation than UNet3.
However, both models fell far short of the ground-truth mask baseline error of approximately \qty{0.25}{\degree}, suggesting that segmentation errors dominate.
Furthermore, our data suggests a correlation between segmentation quality and orientation accuracy.
Nevertheless, even low-mIoU predictions occasionally yielded acceptable orientations.
Many extreme errors appear to stem from mislabelled annotations, suggesting that the true orientation performance may be better than the measurements indicate.

While ResUNet18’s stronger encoder helps in cluttered conditions, it also overfits quickly.
Both models struggle in extreme clutter or when the bee is barely visible.
Potential improvements include better annotations, regularization, and post-processing of predictions to resolve ambiguities.
Finally, leveraging temporal information or multi-frame context, as in the work of Bozek et al.~\cite{bozek2018pixel, bozek2021markerless}, could enhance robustness and orientation estimation further.