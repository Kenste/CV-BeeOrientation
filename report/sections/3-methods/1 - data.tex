We use the Honeybee Segmentation and Tracking Dataset\footnote{Dataset available at \href{https://groups.oist.jp/bptu/honeybee-tracking-dataset}{groups.oist.jp/bptu/honeybee-tracking-dataset}}, introduced by Bozek et al.~\cite{bozek2021markerless}, focusing on the \qty{30}{fps} grayscale video recordings.
This dataset provides annotated frames indicating the positions, orientations, and within-cell states of the bees.

We extract individual training examples from these frames by cropping fully visible bees into 160$\times$160 grayscale images.
The crop size balances two competing goals: ensuring the bee is fully visible and centered, despite size differences and annotation noise, while minimizing the presence of neighboring bees.
Some unavoidable background clutter may appear at the edges of the crop, but the annotated bee is centered and the only segmented individual.
Other bees, if present, are treated as background noise.

For each cropped image, we generate a dense segmentation mask, initialized as background (label~0).
An ellipse is placed at the center of the bee, aligned with the annotated orientation and approximating the bee’s body shape.
The half of the ellipse pointing in the orientation's direction is labeled head (label~1), and the opposite half is labeled tail (label~2).
We also store the ground-truth orientation angle of the bee in a CSV file, along with the corresponding mask and crop filenames.

Processing all annotated frames yields approximately \qty{130266}{} samples for training, validation, and testing.
Occasional annotation errors (e.g., incorrect orientations) were rare and treated as noise that is not expected to bias the training process.

We split the dataset into training (\qty{64}{\percent}), validation (\qty{16}{\percent}), and test (\qty{20}{\percent}) sets.
We ensure reproducibility by fixing and reporting the random seed.